\documentclass[12pt,a4paper]{article}
\usepackage[utf8]{inputenc}
\usepackage[T1]{fontenc}
\usepackage[french]{babel}
\usepackage{geometry}
\usepackage{enumitem}
\usepackage{xcolor}
\usepackage{hyperref}

\geometry{margin=2.5cm}

\title{Cahier des Charges\\
\large Projet 10 : Surveillance automatique de prix et alertes\\
\large PFM – Master SDIA}
\author{}
\date{}

\begin{document}

\maketitle

Oui. Voici un \textbf{cahier des charges complet} pour le \textbf{Projet 10 : Surveillance automatique de prix et alertes} (PFM – Master SDIA).

\vspace{0.5cm}
\rule{\textwidth}{0.4pt}
\vspace{0.5cm}

\section{Contexte \& Problématique}

Les prix en ligne (e-commerce, billets, services, abonnements\ldots) changent fréquemment. Les utilisateurs (clients, responsables achats, vendeurs) ont besoin de :

\begin{itemize}
    \item \textbf{suivre l'évolution} de prix sur une liste de produits,
    \item \textbf{détecter automatiquement} les baisses/hausses significatives,
    \item \textbf{recevoir des alertes} (email / dashboard) au bon moment.
\end{itemize}

\vspace{0.5cm}
\rule{\textwidth}{0.4pt}
\vspace{0.5cm}

\section{Objectifs du projet}

Concevoir une chaîne complète :

\begin{enumerate}
    \item \textbf{Collecte} automatique de prix (web scraping dynamique obligatoire : Selenium/Playwright)
    \item \textbf{Stockage} historique (dataset exploitable, versionné)
    \item \textbf{Analyse \& Data Science} :
    \begin{itemize}
        \item Détection d'anomalies (promotions, hausses suspectes)
        \item Prévision des baisses de prix
    \end{itemize}
    \item \textbf{Système d'alertes} : notifications selon des règles définies
    \item \textbf{Restitution} : dashboard + rapport + soutenance
\end{enumerate}

\vspace{0.5cm}
\rule{\textwidth}{0.4pt}
\vspace{0.5cm}

\section{Périmètre (Scope)}

\subsection{Sources de données (au choix)}

\begin{itemize}
    \item Sites e-commerce (ex : Jumia, Amazon, etc.) \textbf{ou}
    \item Sites locaux/nationaux pertinents (produits tech, électroménager, parapharmacie, etc.)
\end{itemize}

\begin{quote}
Le choix doit respecter les conditions d'utilisation du site et limiter la charge (scraping raisonnable).
\end{quote}

\subsection{Ce qui est inclus}

\begin{itemize}
    \item Scraping \textbf{multi-pages + pagination}
    \item Gestion contenu dynamique (JS, lazy-load)
    \item Nettoyage, déduplication, normalisation devises/format
    \item Historisation (prix par date/heure)
    \item EDA + modèles DS/ML
    \item Alertes \& dashboard
\end{itemize}

\subsection{Ce qui est exclu (par défaut)}

\begin{itemize}
    \item Achat automatisé / contournement agressif anti-bot
    \item Scraping de données privées derrière authentification non autorisée
    \item Extraction massive non contrôlée
\end{itemize}

\vspace{0.5cm}
\rule{\textwidth}{0.4pt}
\vspace{0.5cm}

\section{Utilisateurs cibles}

\begin{itemize}
    \item \textbf{Utilisateur final} : souhaite suivre un panier de produits et recevoir des alertes
    \item \textbf{Analyste} : consulte tendances, anomalies, prévisions
    \item \textbf{Administrateur} : gère la liste des URLs suivies + fréquence de collecte
\end{itemize}

\vspace{0.5cm}
\rule{\textwidth}{0.4pt}
\vspace{0.5cm}

\section{Fonctionnalités attendues}

\subsection{Module ``Configuration de suivi''}

\begin{itemize}
    \item Ajouter / supprimer un produit à suivre via :
    \begin{itemize}
        \item URL produit
        \item nom (optionnel)
        \item seuils d'alerte (ex : baisse > 5\%, prix < X)
        \item fréquence de suivi (ex : 1/jour, 2/jour, etc.)
    \end{itemize}
    \item Regrouper par catégorie / marque
\end{itemize}

\subsection{Module ``Scraping \& Robustesse''}

\begin{itemize}
    \item Extraction minimum par produit :
    \begin{itemize}
        \item \textbf{nom produit}
        \item \textbf{prix}
        \item devise
        \item disponibilité (si possible)
        \item vendeur (si marketplace, option)
        \item date/heure de collecte
        \item URL
    \end{itemize}
    \item Gestion :
    \begin{itemize}
        \item pagination (si liste/catégorie)
        \item timeouts, retries, logs
        \item changements DOM (sélecteurs robustes)
        \item erreurs HTTP / pages indisponibles
    \end{itemize}
    \item Respect des bonnes pratiques : délais entre requêtes, user-agent raisonnable
\end{itemize}

\subsection{Module ``Dataset \& Historique''}

\begin{itemize}
    \item Stockage format \textbf{CSV/Parquet} (minimum exigé) + option DB (SQLite/PostgreSQL)
    \item Schéma de données propre (colonnes normalisées)
    \item Gestion doublons + valeurs manquantes
    \item Versioning (au minimum par date)
\end{itemize}

\subsection{Module ``Analyse Exploratoire (EDA)''}

\begin{itemize}
    \item Statistiques :
    \begin{itemize}
        \item min/max/moyenne, volatilité, variation journalière
        \item distribution des prix par catégorie/marque
    \end{itemize}
    \item Visualisations :
    \begin{itemize}
        \item séries temporelles par produit
        \item boxplots comparatifs
        \item heatmap (corrélations si features)
        \item top baisses / top hausses
    \end{itemize}
\end{itemize}

\subsection{Module ``Détection d'anomalies''}

\begin{itemize}
    \item Définir ``anomalie'' :
    \begin{itemize}
        \item promo exceptionnelle (drop soudain)
        \item hausse anormale
        \item prix incohérent (scraping bug)
    \end{itemize}
    \item Méthodes possibles :
    \begin{itemize}
        \item règles (z-score, IQR)
        \item Isolation Forest / One-Class SVM (bonus)
    \end{itemize}
\end{itemize}

\subsection{Module ``Prévision / ML''}

Objectif : prévoir le prix à court terme \textbf{ou} la probabilité de baisse.

\begin{itemize}
    \item Baselines recommandées :
    \begin{itemize}
        \item moyenne mobile / exponential smoothing
    \end{itemize}
    \item ML (si dataset suffisant) :
    \begin{itemize}
        \item régression (RandomForest/XGBoost/Linear)
        \item modèle de classification ``baisse dans 3 jours : oui/non''
    \end{itemize}
    \item Évaluation :
    \begin{itemize}
        \item split temporel (train $\rightarrow$ test chronologique)
        \item MAE/RMSE (régression), F1/ROC-AUC (classification)
    \end{itemize}
\end{itemize}

\subsection{Module ``Alertes''}

\begin{itemize}
    \item Alertes déclenchées selon :
    \begin{itemize}
        \item prix < seuil
        \item baisse > X\% sur Y jours
        \item anomalie détectée
    \end{itemize}
    \item Canaux :
    \begin{itemize}
        \item email (minimum) \textbf{ou} console + fichier log si email non possible
        \item dashboard (indicateurs ``alerts'')
    \end{itemize}
\end{itemize}

\subsection{Dashboard (Restitution)}

\begin{itemize}
    \item Pages/sections :
    \begin{itemize}
        \item vue globale (kpi : \#produits, \#alertes, plus fortes variations)
        \item recherche + filtre (catégorie, marque)
        \item détail produit (courbe prix + prédiction + anomalies)
    \end{itemize}
    \item Outils :
    \begin{itemize}
        \item Matplotlib/Seaborn (minimum) ou Power BI (option)
        \item Streamlit (bonus très apprécié)
    \end{itemize}
\end{itemize}

\vspace{0.5cm}
\rule{\textwidth}{0.4pt}
\vspace{0.5cm}

\section{Exigences non fonctionnelles}

\subsection{Qualité \& lisibilité du code}

\begin{itemize}
    \item Code structuré : \texttt{scraping/}, \texttt{cleaning/}, \texttt{analysis/}, \texttt{modeling/}, \texttt{alerts/}
    \item Commentaires + README d'exécution
    \item Gestion exceptions + logs
\end{itemize}

\subsection{Performance}

\begin{itemize}
    \item Temps de scraping raisonnable (batch)
    \item Mise en cache / limitation fréquence
    \item Possibilité d'exécuter en ``daily job'' (cron / planificateur)
\end{itemize}

\subsection{Conformité \& éthique}

\begin{itemize}
    \item Respect conditions d'utilisation des sites
    \item Pas de collecte de données personnelles
    \item Débit de requêtes modéré, pas de surcharge
\end{itemize}

\vspace{0.5cm}
\rule{\textwidth}{0.4pt}
\vspace{0.5cm}

\section{Architecture proposée (simple et claire)}

\begin{enumerate}
    \item \textbf{Scraper} (Playwright/Selenium) $\rightarrow$ export brut (raw)
    \item \textbf{Nettoyage} (Pandas) $\rightarrow$ dataset final (Parquet/CSV)
    \item \textbf{Analyse + ML} $\rightarrow$ outputs (figures + métriques)
    \item \textbf{Alerting} $\rightarrow$ email/log + table ``alerts''
    \item \textbf{Dashboard} $\rightarrow$ visualisation + exploration
\end{enumerate}

\vspace{0.5cm}
\rule{\textwidth}{0.4pt}
\vspace{0.5cm}

\section{Données \& format (proposition de schéma)}

Table/CSV \texttt{prices\_history} :

\begin{itemize}
    \item \texttt{timestamp}
    \item \texttt{product\_id} (hash URL)
    \item \texttt{product\_name}
    \item \texttt{price\_value}
    \item \texttt{currency}
    \item \texttt{availability} (option)
    \item \texttt{seller} (option)
    \item \texttt{source\_site}
    \item \texttt{url}
\end{itemize}

Table/CSV \texttt{alerts} :

\begin{itemize}
    \item \texttt{timestamp}
    \item \texttt{product\_id}
    \item \texttt{alert\_type} (threshold\_drop / below\_price / anomaly)
    \item \texttt{message}
    \item \texttt{price\_value}
\end{itemize}

\vspace{0.5cm}
\rule{\textwidth}{0.4pt}
\vspace{0.5cm}

\section{Plan de réalisation (phases)}

\textbf{Phase 1 – Analyse du site} : pages, dynamique, API possible, anti-bot\\
\textbf{Phase 2 – Scraping} : extraction robuste + pagination\\
\textbf{Phase 3 – Nettoyage} : normalisation + dataset final\\
\textbf{Phase 4 – EDA} : stats + graphiques principaux\\
\textbf{Phase 5 – DS/ML} : anomalies + prévision + évaluation\\
\textbf{Phase 6 – Alertes + dashboard}\\
\textbf{Phase 7 – Rapport + soutenance}

\vspace{0.5cm}
\rule{\textwidth}{0.4pt}
\vspace{0.5cm}

\section{Livrables attendus}

Conformément au document :

\begin{enumerate}
    \item \textbf{Code source complet} (.py ou .ipynb)
    \item \textbf{Dataset final} (CSV/Parquet)
    \item \textbf{Rapport scientifique (PDF)} : intro, méthodo, analyse, résultats, limites
    \item \textbf{Présentation orale} 10–15 min
\end{enumerate}

\vspace{0.5cm}
\rule{\textwidth}{0.4pt}
\vspace{0.5cm}

\section{Critères d'acceptation (Checklist)}

\begin{itemize}
    \item Scraper dynamique fonctionne sur au moins \textbf{N produits} (ex : $\geq$ 100)
    \item Historique de prix sur une période (ex : $\geq$ 7 jours ou simulation)
    \item Dataset propre + documentation colonnes
    \item Au moins \textbf{5 visualisations} pertinentes
    \item Détection d'anomalies opérationnelle (avec exemples)
    \item Prévision ou probabilité de baisse avec métriques
    \item Alertes démontrées (email ou logs)
    \item Rapport clair + conclusions critiques
\end{itemize}

\vspace{0.5cm}
\rule{\textwidth}{0.4pt}
\vspace{0.5cm}

Si tu veux, je peux aussi te donner :

\begin{itemize}
    \item une \textbf{structure de dossier} prête à coder,
    \item un \textbf{template de rapport} (plan + sections),
    \item et une \textbf{liste de features ML} pertinentes (promos, volatilité, moyenne mobile, etc.).
\end{itemize}

\end{document}
